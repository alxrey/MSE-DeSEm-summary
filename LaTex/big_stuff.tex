\documentclass[resume]{subfiles}


\begin{document}
\begin{table}[H]
\centering
\begin{tabular}{cll m{5cm} m{5cm}}
N & Nom & unité & description & exemples\\\hline\hline
7 & Application & Données & Utilité pour l'utilisateur (transfert de fichiers, vidéos, etc...) & Web, FTP, IMAP, LDAP, HTTP, SMB\\\hline
6 & Présentation & Données & Formats, mises en formes, cryptage, login & JSON, ASCII, HTML, Unicode\\\hline
5 & Session & Données & Gestion de l'activité & RPC, NetBios\\\hline
4 & Transport & Segments, streams & sous-adressage, communication entre deux processus & TCP, UDP\\\hline
3 & Réseau & Packets & Transport des données dans un réseau maillé & IPv4/IPv6, ARP\\\hline
2 & Liaison & Trame & Adressage local, gestion des erreurs, etc... & Ethernet, CAN, \\\hline
1 & Physique & Bit & Signaux électriques & Wi-Fi, Câble, 1000BASE-T, USB
\end{tabular}
\end{table}
Les couches 1-4 permettent de transférer les données. Les couches 5-7 sont liées à l'utilisation qu'on fait des données.
\end{document}